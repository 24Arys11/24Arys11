%========================================
%  C O P Y R I G H T
%-------------------------------------------------------------------------------
% @copyright (c) 2023 - 2025 by Dima Darius. All rights reserved.
%
% The reproduction, distribution and utilization of this material 
% without express authorization is prohibited. Offenders will be 
% held liable for the payment of damages. All rights reserved.
%-------------------------------------------------------------------------------
% For permission to use, copy, modify, or distribute this template 
% for any purpose, please contact darius.dimaa@gmail.com.
%========================================

%	Ă		Â		Î		Ș		Ț		ă		â		î		ș		ț
%	\u{A}		\^{A}		\^{I}		\c{S}		\c{T}		\u{a}		\^{a}		\^{i}		\c{s}		\c{t}

\section{Professional experience}

\entry{2022 -- Present}{Cluj-Napoca, RO}{Software Engineer}{Robert Bosch Engineering Center}	% time, place, role, institution,
{
\idea Built and rolled out an \emph{AI Code Reviewer} for our codebase; reduced manual review overhead and accelerated delivery for developers.
\idea Contributed in \emph{Generative AI for Software Development} communities (20\% time); advocated \emph{GitHub Copilot} and agentic workflows.
\idea Delivered features across the web stack: \emph{Angular} (frontend), \emph{C\#/.NET} APIs, \emph{MongoDB}.
\idea TesserHub accelerator: validated and implemented a new automotive safety function; led an innovation initiative in my department.
\idea Completed the \emph{Hugging Face Agents Course}; built a local-LLM agent with \emph{Qwen-32B} using \emph{LangGraph} and \emph{LlamaIndex} to tackle complex domain questions.
\idea Published “\emph{Introduction to RAG Systems}” in Today’s Software Magazine (TSM); open-sourced an end-to-end \emph{RAG} implementation in Python (see \href{https://github.com/24Arys11/RAG}{GitHub}).
\idea Agile, cross-functional development for safety-critical automotive software in a \emph{multinational} team.
}